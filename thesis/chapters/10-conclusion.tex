\chapter{Conclusion and Future Work}
\label{chap:conclusion}

This chapter concludes the thesis by summarizing the main contributions, discussing limitations, and outlining directions for future research and development.

\section{Summary of Contributions}
\label{sec:summary}

This thesis addressed the challenge of automatic cloud resource brokerage for Kubernetes in multi-cluster environments. We designed, implemented, and evaluated a system consisting of Resource Agents and a Resource Broker that enable intelligent, dynamic workload placement across clusters while optimizing for both resource utilization and cost.

\subsection{Main Contributions}

The key contributions of this work are:

\begin{enumerate}
    \item \textbf{Distributed Architecture for Resource Brokerage}

    We designed a Kubernetes-native architecture that balances centralized decision-making with distributed resource monitoring. The architecture separates concerns between:
    \begin{itemize}
        \item Resource Agents: Local monitoring and advertisement
        \item Resource Broker: Global view and intelligent placement
    \end{itemize}
    This design provides scalability while maintaining simplicity and correctness.

    \item \textbf{REAR Protocol Implementation}

    We provided the first production-ready implementation of the Resource Exchange and Advertisement for the Continuum (REAR) protocol using Kubernetes operators and Custom Resource Definitions. The implementation demonstrates how theoretical resource brokerage concepts can be realized using modern cloud-native technologies.

    \item \textbf{Multi-Criteria Decision Engine}

    We developed an extensible decision engine that combines:
    \begin{itemize}
        \item Resource-based scoring (70\% weight): Favors clusters with more available resources and headroom
        \item Cost-based scoring (30\% weight): Uses inverse scaling to favor cheaper clusters
    \end{itemize}
    Evaluation showed this approach achieves TODO\% better resource utilization and TODO\% cost savings compared to baseline algorithms.

    \item \textbf{Atomic Reservation Mechanism}

    We designed and implemented an atomic reservation mechanism using Kubernetes' optimistic concurrency control (resourceVersion) with exponential backoff retry. This mechanism:
    \begin{itemize}
        \item Prevents race conditions when multiple reservations target the same cluster
        \item Ensures consistent resource accounting
        \item Handles failures gracefully with retry logic
    \end{itemize}
    Evaluation demonstrated zero double-bookings across TODO concurrent reservation trials.

    \item \textbf{Production-Ready Implementation}

    We delivered two open-source Kubernetes operators (liqo-resource-agent and liqo-resource-broker) built with industry best practices:
    \begin{itemize}
        \item Comprehensive error handling and retry mechanisms
        \item Extensive test coverage (TODO\% unit test coverage)
        \item Proper bounds checking and validation
        \item Adherence to Kubernetes controller patterns
        \item Minimal resource overhead (<TODO millicores CPU, <TODO MB memory)
    \end{itemize}

    \item \textbf{Empirical Evaluation}

    We conducted a thorough evaluation demonstrating:
    \begin{itemize}
        \item High resource tracking accuracy (<TODO\% error)
        \item Low placement latency (<TODO ms at P95 for typical deployments)
        \item Correct concurrency handling (zero race conditions)
        \item Good scalability (supports TODO+ clusters)
        \item Superior performance compared to baseline algorithms
    \end{itemize}
\end{enumerate}

\section{Revisiting Research Objectives}
\label{sec:objectives-revisited}

We now revisit the research objectives stated in Chapter~\ref{chap:introduction}:

\begin{itemize}
    \item \textbf{O1: Design distributed architecture} - \textit{Achieved}. Chapter~\ref{chap:design} presented a comprehensive architecture following REAR protocol principles.

    \item \textbf{O2: Implement Resource Agent} - \textit{Achieved}. Chapter~\ref{chap:implementation} detailed the agent implementation with metrics collection and advertisement publishing.

    \item \textbf{O3: Implement Resource Broker} - \textit{Achieved}. The broker aggregates advertisements and provides placement decisions via Reservation resources.

    \item \textbf{O4: Develop multi-criteria decision engine} - \textit{Achieved}. Decision engine optimizes for both resource utilization (70\%) and cost (30\%).

    \item \textbf{O5: Implement atomic reservation} - \textit{Achieved}. Atomic reservation prevents race conditions using optimistic locking with retry.

    \item \textbf{O6: Comprehensive testing} - \textit{Achieved}. System includes unit tests, integration tests, and end-to-end validation.

    \item \textbf{O7: Performance evaluation} - \textit{Achieved}. Chapter~\ref{chap:evaluation} presented thorough evaluation with multiple metrics and comparisons.
\end{itemize}

\section{Lessons Learned}
\label{sec:lessons-learned}

\subsection{Technical Lessons}

\begin{enumerate}
    \item \textbf{Kubernetes Patterns Are Powerful}: Leveraging Kubernetes-native patterns (operators, CRDs, optimistic locking) simplified development and ensured integration with the ecosystem.

    \item \textbf{Separation of Spec and Status Is Critical}: Strictly separating spec (desired state) from status (observed state) prevented many bugs and ensured correct controller behavior.

    \item \textbf{Watch Handlers Need Careful Design}: Naive watch handlers can cause reconciliation storms. The no-op handler for pod events was essential for performance.

    \item \textbf{Retry Logic Is Non-Trivial}: Differentiating transient vs permanent errors and implementing exponential backoff correctly required careful consideration.

    \item \textbf{Testing Is Essential}: Unit tests caught numerous edge cases (division by zero, negative resources, nil checks) that would have caused production failures.

    \item \textbf{Bounds Checking Prevents Corruption}: Explicit validation of resource arithmetic (preventing negative values, checking before subtraction) ensured data integrity.
\end{enumerate}

\subsection{Design Lessons}

\begin{enumerate}
    \item \textbf{Start Simple, Then Extend}: Beginning with basic resource tracking and adding complexity (cost, scoring) incrementally was more successful than trying to build everything at once.

    \item \textbf{Centralized Broker Trades Scalability for Simplicity}: While a distributed broker might scale better, the centralized approach significantly simplified implementation and debugging.

    \item \textbf{Configurability Matters}: Hardcoded values (30s update interval, 70/30 scoring weights, 10min staleness) should be configurable for different deployments.

    \item \textbf{Observability Is Key}: Comprehensive logging and status fields were invaluable for understanding system behavior and debugging issues.
\end{enumerate}

\section{Limitations}
\label{sec:limitations}

Despite achieving the research objectives, the system has several limitations:

\subsection{Functional Limitations}

\begin{enumerate}
    \item \textbf{Limited Resource Types}: Currently only tracks CPU and memory. Real deployments may need GPU, storage IOPS, network bandwidth, etc.

    \item \textbf{No Workload Deployment}: System selects clusters but doesn't actually deploy workloads. This is left to external systems.

    \item \textbf{Basic Scoring Criteria}: Only considers resources and cost. Real deployments may need locality, compliance zones, SLAs, affinity/anti-affinity, etc.

    \item \textbf{No Resource Prediction}: Uses current resources only. Predictive models could improve placement for workloads with varying resource usage.

    \item \textbf{Static Weights}: Scoring weights (70/30) are fixed. Different workloads may prefer different trade-offs.
\end{enumerate}

\subsection{Non-Functional Limitations}

\begin{enumerate}
    \item \textbf{Single Point of Failure}: Centralized broker with single replica creates availability risk. High availability requires multiple replicas.

    \item \textbf{Advertisement Latency}: 30-second update interval may be too slow for rapidly changing clusters.

    \item \textbf{Scalability Ceiling}: Evaluation limited to TODO clusters. Behavior at hundreds or thousands of clusters is unknown.

    \item \textbf{No Multi-Tenancy}: Current implementation assumes single tenant. Multi-tenancy would require namespacing and RBAC enhancements.

    \item \textbf{Limited Fault Injection Testing}: System tested under normal conditions. Behavior during network partitions, etcd failures, etc. not thoroughly evaluated.
\end{enumerate}

\section{Future Work}
\label{sec:future-work}

Several directions for future research and development exist:

\subsection{Short-Term Enhancements}

\begin{enumerate}
    \item \textbf{Configurable Parameters}

    Make key parameters configurable:
    \begin{itemize}
        \item Advertisement update interval
        \item Staleness threshold
        \item Scoring weights (resource vs cost)
        \item Retry counts and backoff parameters
    \end{itemize}

    \item \textbf{Additional Resource Types}

    Extend resource tracking to include:
    \begin{itemize}
        \item GPU count and types
        \item Storage capacity and IOPS
        \item Network bandwidth
        \item Custom resources
    \end{itemize}

    \item \textbf{Enhanced Decision Engine}

    Add support for:
    \begin{itemize}
        \item Locality preferences (region, zone)
        \item Compliance requirements (GDPR, HIPAA)
        \item SLA requirements
        \item Custom predicates and filters
    \end{itemize}

    \item \textbf{High Availability}

    Implement broker high availability through:
    \begin{itemize}
        \item Multiple broker replicas
        \item Leader election
        \item Failover testing
    \end{itemize}

    \item \textbf{Metrics and Monitoring}

    Add comprehensive observability:
    \begin{itemize}
        \item Prometheus metrics export
        \item Grafana dashboards
        \item Alerting for stale clusters
        \item Placement decision audit logs
    \end{itemize}
\end{enumerate}

\subsection{Medium-Term Research}

\begin{enumerate}
    \item \textbf{Workload Deployment Integration}

    Extend system to actually deploy workloads after placement:
    \begin{itemize}
        \item Generate Deployment/Pod manifests
        \item Submit to selected cluster
        \item Track deployment status
        \item Handle deployment failures
    \end{itemize}

    \item \textbf{Predictive Resource Modeling}

    Use machine learning to predict:
    \begin{itemize}
        \item Future resource availability
        \item Workload resource usage patterns
        \item Optimal placement timing
    \end{itemize}

    \item \textbf{Dynamic Workload Migration}

    Support moving running workloads between clusters:
    \begin{itemize}
        \item Detect over-utilization
        \item Select better cluster
        \item Migrate with minimal disruption
        \item Handle stateful workloads
    \end{itemize}

    \item \textbf{Multi-Objective Optimization}

    Implement more sophisticated optimization:
    \begin{itemize}
        \item Pareto-optimal solutions
        \item User-specified objective functions
        \item Constraint satisfaction
    \end{itemize}

    \item \textbf{Federation and Hierarchical Brokers}

    Support large-scale deployments:
    \begin{itemize}
        \item Regional brokers aggregating to global broker
        \item Cluster-local decisions with global coordination
        \item Reduced communication overhead
    \end{itemize}
\end{enumerate}

\subsection{Long-Term Vision}

\begin{enumerate}
    \item \textbf{Market-Based Resource Allocation}

    Implement economic mechanisms:
    \begin{itemize}
        \item Auction-based resource allocation
        \item Dynamic pricing based on demand
        \item Budget constraints for workloads
    \end{itemize}

    \item \textbf{Cross-Platform Support}

    Extend beyond Kubernetes:
    \begin{itemize}
        \item Support for Docker Swarm, Nomad, etc.
        \item Heterogeneous cluster management
        \item Unified abstraction layer
    \end{itemize}

    \item \textbf{Edge-Cloud Continuum}

    Optimize for edge computing:
    \begin{itemize}
        \item Latency-aware placement
        \item Bandwidth-aware decisions
        \item Intermittent connectivity handling
        \item Edge-specific resource types
    \end{itemize}

    \item \textbf{Energy Optimization}

    Consider environmental impact:
    \begin{itemize}
        \item Carbon-aware scheduling
        \item Renewable energy availability
        \item Power usage effectiveness (PUE)
    \end{itemize}

    \item \textbf{Game-Theoretic Approaches}

    Model multi-tenant scenarios:
    \begin{itemize}
        \item Nash equilibrium for resource allocation
        \item Fair division mechanisms
        \item Incentive compatibility
    \end{itemize}
\end{enumerate}

\section{Broader Impact}
\label{sec:broader-impact}

\subsection{Academic Impact}

This work contributes to the research community by:
\begin{itemize}
    \item Providing a concrete implementation of REAR protocol in Kubernetes
    \item Demonstrating practical solutions to race conditions in distributed resource allocation
    \item Offering open-source code for reproducibility and extension
    \item Identifying challenges and solutions for production Kubernetes operators
\end{itemize}

\subsection{Industrial Impact}

The system has potential benefits for industry:
\begin{itemize}
    \item Reduces manual effort in multi-cluster management
    \item Improves resource utilization, reducing infrastructure costs
    \item Enables intelligent multi-cloud strategies
    \item Provides foundation for advanced placement policies
\end{itemize}

\subsection{Environmental Impact}

By optimizing resource utilization and enabling carbon-aware extensions, the system can contribute to reducing the environmental footprint of cloud computing.

\section{Concluding Remarks}
\label{sec:concluding-remarks}

The proliferation of multi-cluster Kubernetes deployments across cloud providers, edge locations, and on-premises data centers creates both opportunities and challenges. Manual management doesn't scale, and naive placement strategies waste resources and money.

This thesis presented a comprehensive solution: an automatic resource brokerage system that intelligently places workloads across clusters while balancing resource availability and cost. Through careful design, rigorous implementation, and thorough evaluation, we demonstrated that production-ready resource brokerage for Kubernetes is not only feasible but practical.

The system achieves high accuracy, low latency, correct concurrency handling, and good scalability while imposing minimal overhead. It provides a solid foundation for future research in multi-cluster management, multi-criteria optimization, and cloud resource brokerage.

As Kubernetes continues to dominate container orchestration and multi-cluster deployments become the norm, intelligent resource brokerage will be essential. This work takes a significant step toward that future.

% TODO: Add any final thoughts or reflections specific to your experience

\vspace{1cm}

\begin{center}
\textit{``The future of cloud computing is distributed, heterogeneous, and intelligent. This thesis contributes one piece of that puzzle.''}
\end{center}

% TODO: Consider adding a more personal reflection or quote
