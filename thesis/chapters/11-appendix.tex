\chapter{Appendix}
\label{chap:appendix}

\section{Installation Guide}
\label{sec:installation}

\subsection{Prerequisites}

Before installing the Resource Agent and Resource Broker, ensure you have:

\begin{itemize}
    \item Kubernetes cluster(s) v1.24+
    \item kubectl configured with cluster access
    \item Sufficient RBAC permissions to install CRDs and operators
    \item (Optional) Helm v3+ for simplified deployment
\end{itemize}

\subsection{Installing Resource Broker}

\begin{enumerate}
    \item Clone the repository:
\begin{lstlisting}[language=bash]
git clone https://github.com/MehdiAzizian/liqo-resource-broker.git
cd liqo-resource-broker
\end{lstlisting}

    \item Install CRDs:
\begin{lstlisting}[language=bash]
make install
\end{lstlisting}

    \item Deploy the broker:
\begin{lstlisting}[language=bash]
make deploy IMG=<your-image-registry>/liqo-resource-broker:latest
\end{lstlisting}

    \item Verify deployment:
\begin{lstlisting}[language=bash]
kubectl get pods -n liqo-resource-broker-system
kubectl get crds | grep broker.fluidos.eu
\end{lstlisting}
\end{enumerate}

\subsection{Installing Resource Agent}

\begin{enumerate}
    \item Clone the repository:
\begin{lstlisting}[language=bash]
git clone https://github.com/MehdiAzizian/liqo-resource-agent.git
cd liqo-resource-agent
\end{lstlisting}

    \item Install CRDs:
\begin{lstlisting}[language=bash]
make install
\end{lstlisting}

    \item Configure broker connection (edit config/manager/manager.yaml):
\begin{lstlisting}[language=yaml]
env:
- name: BROKER_ENDPOINT
  value: "<broker-cluster-api-endpoint>"
- name: BROKER_TOKEN
  value: "<service-account-token>"
\end{lstlisting}

    \item Deploy the agent:
\begin{lstlisting}[language=bash]
make deploy IMG=<your-image-registry>/liqo-resource-agent:latest
\end{lstlisting}

    \item Verify deployment:
\begin{lstlisting}[language=bash]
kubectl get pods -n liqo-resource-agent-system
kubectl get advertisements
\end{lstlisting}
\end{enumerate}

\subsection{Creating a Reservation}

\begin{lstlisting}[language=yaml, caption={Example reservation}, label={lst:example-reservation}]
apiVersion: broker.fluidos.eu/v1alpha1
kind: Reservation
metadata:
  name: example-workload
  namespace: default
spec:
  targetClusterID: ""  # Empty for automatic selection
  requestedResources:
    cpu: "4"
    memory: "8Gi"
  requirements:
    maxCost: "0.10"
\end{lstlisting}

Apply the reservation:
\begin{lstlisting}[language=bash]
kubectl apply -f reservation.yaml
\end{lstlisting}

Check the status:
\begin{lstlisting}[language=bash]
kubectl get reservation example-workload -o yaml
\end{lstlisting}

\section{Configuration Reference}
\label{sec:config-reference}

\subsection{Resource Agent Configuration}

\begin{table}[ht]
\centering
\caption{Resource Agent configuration parameters}
\label{tab:ra-config}
\begin{tabular}{@{}lll@{}}
\toprule
\textbf{Parameter} & \textbf{Description} & \textbf{Default} \\ \midrule
CLUSTER\_ID & Unique identifier for this cluster & auto-detected \\
BROKER\_ENDPOINT & Broker cluster API server URL & Required \\
BROKER\_TOKEN & Authentication token for broker & Required \\
UPDATE\_INTERVAL & Advertisement update interval & 30s \\
NAMESPACE & Namespace for Advertisement resource & default \\
ENABLE\_COST & Include cost information & false \\
COST\_AMOUNT & Cost per CPU-hour & "0.05" \\ \bottomrule
\end{tabular}
\end{table}

\subsection{Resource Broker Configuration}

\begin{table}[ht]
\centering
\caption{Resource Broker configuration parameters}
\label{tab:rb-config}
\begin{tabular}{@{}lll@{}}
\toprule
\textbf{Parameter} & \textbf{Description} & \textbf{Default} \\ \midrule
STALENESS\_THRESHOLD & Time before marking advertisement stale & 10m \\
RESOURCE\_WEIGHT & Weight for resource scoring (0-1) & 0.7 \\
COST\_WEIGHT & Weight for cost scoring (0-1) & 0.3 \\
MAX\_RETRIES & Max retries for atomic reservation & 5 \\
BASE\_BACKOFF & Initial backoff for retries & 100ms \\ \bottomrule
\end{tabular}
\end{table}

\section{API Reference}
\label{sec:api-reference}

\subsection{Advertisement Resource}

\textbf{API Group}: rear.fluidos.eu/v1alpha1

\textbf{Kind}: Advertisement

\textbf{Spec Fields}:
\begin{itemize}
    \item \texttt{clusterID} (string): Unique cluster identifier
    \item \texttt{timestamp} (metav1.Time): When metrics were collected
    \item \texttt{resources} (ResourceMetrics): Resource information
    \begin{itemize}
        \item \texttt{allocatable} (ResourceQuantities): Total allocatable resources
        \item \texttt{allocated} (ResourceQuantities): Currently allocated resources
        \item \texttt{available} (ResourceQuantities): Currently available resources
    \end{itemize}
    \item \texttt{cost} (CostInformation, optional): Cost information
    \begin{itemize}
        \item \texttt{amount} (string): Cost value
        \item \texttt{currency} (string): Currency code (e.g., "USD")
        \item \texttt{unit} (string): Unit of measurement (e.g., "cpu-hour")
    \end{itemize}
\end{itemize}

\textbf{Status Fields}:
\begin{itemize}
    \item \texttt{published} (bool): Whether successfully published to broker
    \item \texttt{lastPublishTime} (metav1.Time): Last publish timestamp
    \item \texttt{message} (string): Human-readable status message
\end{itemize}

\subsection{ClusterAdvertisement Resource}

\textbf{API Group}: broker.fluidos.eu/v1alpha1

\textbf{Kind}: ClusterAdvertisement

\textbf{Spec Fields}: Same as Advertisement, plus:
\begin{itemize}
    \item \texttt{resources.reserved} (ResourceQuantities): Reserved resources
\end{itemize}

\textbf{Status Fields}:
\begin{itemize}
    \item \texttt{active} (bool): Whether advertisement is fresh (not stale)
    \item \texttt{phase} (string): Current phase ("Active", "Stale")
    \item \texttt{score} (string): Computed score for this cluster
    \item \texttt{lastUpdateTime} (metav1.Time): Last status update
    \item \texttt{message} (string): Human-readable status message
\end{itemize}

\subsection{Reservation Resource}

\textbf{API Group}: broker.fluidos.eu/v1alpha1

\textbf{Kind}: Reservation

\textbf{Spec Fields}:
\begin{itemize}
    \item \texttt{targetClusterID} (string): Specific cluster ID (empty for auto-select)
    \item \texttt{requestedResources} (ResourceQuantities): Required resources
    \item \texttt{requirements} (ReservationRequirements, optional): Additional constraints
\end{itemize}

\textbf{Status Fields}:
\begin{itemize}
    \item \texttt{phase} (string): Current phase ("Pending", "Reserved", "Active", "Failed", "Completed")
    \item \texttt{selectedClusterID} (string): Cluster where resources were reserved
    \item \texttt{reservationTime} (metav1.Time): When reservation was made
    \item \texttt{message} (string): Human-readable status message
\end{itemize}

\section{Common Issues and Troubleshooting}
\label{sec:troubleshooting}

\subsection{Agent Cannot Connect to Broker}

\textbf{Symptoms}: Advertisement status shows \texttt{published: false} with message about connection failure.

\textbf{Possible Causes}:
\begin{itemize}
    \item Incorrect BROKER\_ENDPOINT configuration
    \item Invalid or expired BROKER\_TOKEN
    \item Network connectivity issues
    \item Broker not running or CRDs not installed
\end{itemize}

\textbf{Solutions}:
\begin{enumerate}
    \item Verify broker endpoint is correct and accessible
    \item Check broker pod logs: \texttt{kubectl logs -n liqo-resource-broker-system <pod>}
    \item Verify service account token has proper RBAC permissions
    \item Test connectivity from agent cluster to broker cluster
\end{enumerate}

\subsection{Reservation Stuck in Pending}

\textbf{Symptoms}: Reservation remains in "Pending" phase indefinitely.

\textbf{Possible Causes}:
\begin{itemize}
    \item No clusters have sufficient resources
    \item All advertisements are stale
    \item Reservation controller not running
    \item Decision engine error
\end{itemize}

\textbf{Solutions}:
\begin{enumerate}
    \item Check ClusterAdvertisement resources: \texttt{kubectl get clusteradvertisements}
    \item Verify at least one is Active with sufficient resources
    \item Check broker controller logs for errors
    \item Reduce requested resources if too high
\end{enumerate}

\subsection{Resource Tracking Inaccuracy}

\textbf{Symptoms}: Advertised resources don't match actual cluster state.

\textbf{Possible Causes}:
\begin{itemize}
    \item Pending pods counted as allocated
    \item Failed pods not filtered out
    \item Not-ready nodes counted as allocatable
    \item Reserved resources not properly updated
\end{itemize}

\textbf{Solutions}:
\begin{enumerate}
    \item Check agent logs for errors in metrics collection
    \item Verify node readiness states
    \item Manually verify allocated resources match pod requests
    \item Check for completed/failed reservations not cleaned up
\end{enumerate}

\section{Performance Tuning}
\label{sec:performance-tuning}

\subsection{Reducing Agent Overhead}

\begin{itemize}
    \item Increase UPDATE\_INTERVAL to reduce API server load (trade-off: less fresh data)
    \item Use client caching (enabled by default)
    \item Filter nodes by labels if only subset should be monitored
    \item Exclude certain namespaces from pod counting
\end{itemize}

\subsection{Reducing Broker Latency}

\begin{itemize}
    \item Deploy broker on high-performance nodes
    \item Use SSD storage for etcd
    \item Increase broker replica count for horizontal scaling
    \item Implement caching for frequently accessed ClusterAdvertisements
\end{itemize}

\subsection{Handling High Concurrency}

\begin{itemize}
    \item Increase MAX\_RETRIES if seeing many conflicts
    \item Adjust BASE\_BACKOFF and backoff factor
    \item Implement request queuing to serialize high-contention scenarios
    \item Consider sharding reservations across multiple brokers
\end{itemize}

\section{Code Repository Structure}
\label{sec:repo-structure}

\subsection{Resource Agent Repository}

\begin{lstlisting}
liqo-resource-agent/
├── api/v1alpha1/              # CRD types
├── config/                    # Kubernetes manifests
│   ├── crd/                   # CRD YAML
│   ├── manager/               # Deployment
│   ├── rbac/                  # RBAC rules
│   └── samples/               # Example resources
├── internal/
│   ├── controller/            # Controllers
│   ├── metrics/               # Metrics collection
│   └── publisher/             # Broker publishing
├── cmd/main.go
├── Dockerfile
├── Makefile
└── README.md
\end{lstlisting}

\subsection{Resource Broker Repository}

\begin{lstlisting}
liqo-resource-broker/
├── api/v1alpha1/              # CRD types
├── config/                    # Kubernetes manifests
├── internal/
│   ├── broker/                # Decision engine
│   ├── controller/            # Controllers
│   └── resource/              # Resource calculations
├── test/
│   ├── e2e/                   # End-to-end tests
│   └── utils/                 # Test utilities
├── cmd/main.go
├── Dockerfile
├── Makefile
└── README.md
\end{lstlisting}

\section{Experimental Data}
\label{sec:experimental-data}

% TODO: Include raw experimental data, graphs, or supplementary tables that didn't fit in the main evaluation chapter

\subsection{Detailed Latency Measurements}

% TODO: Add detailed latency breakdown tables

\subsection{Resource Utilization Timeseries}

% TODO: Add graphs showing utilization over time for case study

\subsection{Cost Analysis}

% TODO: Add detailed cost breakdown comparing algorithms

\section{Published Papers and Code}
\label{sec:publications}

\subsection{Source Code Repositories}

\begin{itemize}
    \item Resource Agent: \url{https://github.com/MehdiAzizian/liqo-resource-agent}
    \item Resource Broker: \url{https://github.com/MehdiAzizian/liqo-resource-broker}
\end{itemize}

% TODO: Add DOI links if you archive code in Zenodo or similar

\subsection{Related Publications}

% TODO: List any papers, posters, or demos based on this work

% If you publish workshop papers, conference papers, or demos based on this thesis, list them here:
% \begin{itemize}
%     \item Author Name, "Paper Title," Conference/Workshop, Year.
% \end{itemize}

\section{Acknowledgments for External Code}
\label{sec:code-acknowledgments}

This project builds upon and uses code from the following open-source projects:

\begin{itemize}
    \item \textbf{Kubernetes}: \url{https://kubernetes.io/} (Apache License 2.0)
    \item \textbf{Kubebuilder}: \url{https://kubebuilder.io/} (Apache License 2.0)
    \item \textbf{controller-runtime}: \url{https://github.com/kubernetes-sigs/controller-runtime} (Apache License 2.0)
    \item \textbf{Liqo}: \url{https://liqo.io/} (Apache License 2.0) - Inspiration for architecture
    \item \textbf{Ginkgo/Gomega}: \url{https://onsi.github.io/ginkgo/} (MIT License)
\end{itemize}

% TODO: Add specific acknowledgments if you used code snippets from Stack Overflow, blog posts, etc.
